\documentclass[12pt]{beamer}
\usetheme{Madrid} % Vous pouvez choisir un autre thème Beamer

\usepackage[utf8]{inputenc}
\usepackage[T1]{fontenc}
\usepackage[french]{babel}
\usepackage{graphicx}
\usepackage{ragged2e} % For \RaggedRight

% Titre, Auteur, etc.
\title{Orchestration de Micro-VM Firecracker via un Résolveur DNS}
\author{Efe ERKEN}
\institute{Master Informatique - Parcours SAR \\ Sorbonne Université}
\date{\today}

% ----- Personnalisation du pied de page -----
\setbeamertemplate{footline}{%
	\leavevmode%
	\hbox{%
		\begin{beamercolorbox}[wd=.2\paperwidth,ht=2.25ex,dp=1ex,left]{author in head/foot}%
			\usebeamerfont{author in head/foot}\hspace*{2ex}Efe ERKEN % Votre nom ici
		\end{beamercolorbox}%
		\begin{beamercolorbox}[wd=.6\paperwidth,ht=2.25ex,dp=1ex,center]{title in head/foot}%
			\usebeamerfont{title in head/foot}\insertshorttitle % Utilise le titre court défini dans \title[]{}
		\end{beamercolorbox}%
		\begin{beamercolorbox}[wd=.2\paperwidth,ht=2.25ex,dp=1ex,right]{date in head/foot}%
			\usebeamerfont{date in head/foot}\insertframenumber{} / \inserttotalframenumber\hspace*{2ex} % Numéro de page
	\end{beamercolorbox}}%
	\vskip0pt%
}
% ----- Fin de la personnalisation -----

\begin{document}

	% Page de titre
	\begin{frame}
		\titlepage
	\end{frame}

	% Table des matières (optionnel)
	\begin{frame}
	  \frametitle{Table des Matières}
	  \tableofcontents
	\end{frame}

	% Introduction et Objectifs
    \section{Introduction et Objectifs du Projet}
	\begin{frame}
		\frametitle{Introduction et Objectifs du Projet}
		\begin{itemize}
			\item \textbf{Objectif principal :} Gestion transparente de microVM Firecracker via DNS pour l'exécution sécurisée et isolée de fonctions utilisateurs.
			\item \textbf{Fonctionnement envisagé :}
			\begin{itemize}
				\item Un proxy DNS intercepte les requêtes.
				\item Si le domaine correspond à un service géré, une microVM est créée/allouée.
				\item L'adresse IP de la microVM est retournée dans la réponse DNS.
				\item L'utilisateur accède à la fonction via HTTP sur cette IP.
			\end{itemize}
			\item \textbf{Extensions futures (initialement prévues) :} Réutilisation des microVM, pool d'instances, chaînage de fonctions.
		\end{itemize}
	\end{frame}

	% Architecture microDNS
    \section{Architecture du Gestionnaire \texttt{microDNS}}
	\begin{frame}
		\frametitle{Architecture du Gestionnaire \texttt{microDNS}}
		\begin{columns}[T] % Split frame into columns
			\begin{column}{0.5\textwidth}
				\RaggedRight % Justify text to the left
				\textbf{Composants Clés :}
				\begin{itemize}
					\item \texttt{QueryReceiver (QR)} : Réceptionne et filtre les requêtes DNS.
					\item \texttt{QueryForwarder (QF)} : Relaye les requêtes DNS non gérées vers des résolveurs externes.
					\item \texttt{MicroVMManager (MVMM)} : (Prévu) Gère le cycle de vie des microVM.
				\end{itemize}
			\end{column}
			\begin{column}{0.5\textwidth}
				\begin{figure}
					\includegraphics[width=0.7\linewidth]{microDNS core.png}
					\caption{Architecture et placement de \texttt{microDNS}}
				\end{figure}
			\end{column}
		\end{columns}
	\end{frame}

	% Architecture microDNS
	\begin{frame}
		\frametitle{Architecture du Gestionnaire \texttt{microDNS}}
		\begin{columns}[T] % Split frame into columns
			\begin{column}{0.5\textwidth}
				\RaggedRight % Justify text to the left
				\textbf{Positionnement :}
				\begin{itemize}
					\item Intercepte les requêtes DNS très tôt dans la chaîne du système local.
					\item Remplace \texttt{/etc/resolv.conf} pour pointer vers \texttt{microDNS} (ex: 127.0.0.1:53).
					\item Gère la complexité introduite par des services comme \texttt{systemd-resolved}.
				\end{itemize}
			\end{column}
			\begin{column}{0.5\textwidth}
				\begin{figure}
					\includegraphics[width=0.7\linewidth]{microDNS core.png}
					\caption{Architecture et placement de \texttt{microDNS}}
				\end{figure}
			\end{column}
		\end{columns}
	\end{frame}

	% Architecture microDNS
	\begin{frame}
		\frametitle{Architecture du Gestionnaire \texttt{microDNS}}
				\begin{figure}
					\centering
					\includegraphics[width=0.4\linewidth]{microDNS core.png}
					\caption{Architecture et placement de \texttt{microDNS}}
				\end{figure}
	\end{frame}

	% Architecture MicroVM et Firecracker
    \section{Architecture Interne des MicroVM Firecracker}
	\begin{frame}
		\frametitle{Architecture Interne des MicroVM Firecracker}
		\begin{columns}[T]
			\begin{column}{0.5\textwidth}
				\RaggedRight
				Chaque instance Firecracker exécute une microVM isolée.

				\textbf{Composants internes à la \\ microVM :}
				\begin{itemize}
					\item Serveur web (Flask) : Exécute la fonction utilisateur demandée via HTTP et retourne le résultat.
					\item Exécutables des fonctions utilisateurs.
					\item Programme de communication avec \texttt{microDNS} (via \texttt{Vsock}, non implémenté).
				\end{itemize}
			\end{column}
			\begin{column}{0.48\textwidth}
				\begin{figure}
					\includegraphics[width=\linewidth]{microVM core.png}
					\caption{Architecture interne d'une microVM}
				\end{figure}
			\end{column}
		\end{columns}
	\end{frame}

	% Architecture MicroVM et Firecracker
	\begin{frame}
		\frametitle{Architecture Interne des MicroVM Firecracker}
		\begin{columns}[T]
			\begin{column}{0.5\textwidth}
				\RaggedRight
				\textbf{Communication interne et externe :}
				\begin{itemize}
					\item \texttt{Vsock} : Communication directe avec le gestionnaire \texttt{microDNS} sur l'hôte (contourne la pile réseau).
					\item FIFO POSIX : Communication entre le serveur web et le programme \texttt{Vsock} pour instructions
					\item Messages "Heartbeat" : Envoyés à \texttt{microDNS} pour monitoring.
				\end{itemize}
			\end{column}
			\begin{column}{0.5\textwidth}
				\begin{figure}
					\includegraphics[width=\linewidth]{microVM core.png}
					\caption{Architecture interne d'une microVM}
				\end{figure}
			\end{column}
		\end{columns}
	\end{frame}

	% Architecture MicroVM et Firecracker
	\begin{frame}
		\frametitle{Architecture Interne des MicroVM Firecracker}
				\begin{figure}
					\centering
					\includegraphics[width=0.7\linewidth]{microVM core.png}
					\caption{Architecture interne d'une microVM}
				\end{figure}
	\end{frame}

	% Choix d'Implémentation
    \section{Choix d'Implémentation Majeurs}
	\begin{frame}
		\frametitle{Choix d'Implémentation Majeurs}
		\textbf{\texttt{microDNS} (Proxy et Gestionnaire DNS)}
		\begin{itemize}
			\item \textbf{Langage :} C++
			\begin{itemize}
				\item Contrôle bas niveau (sockets, paquets DNS, interfaces TAP).
				\item Facilités haut niveau (STL, POO, threads) pour un développement accéléré.
			\end{itemize}
			\item \textbf{Classes clés implémentées :}
			\begin{itemize}
				\item \texttt{ConfigLoader} : Chargement de configuration (\texttt{./microDNS.conf}).
				\item \texttt{MessageQueue} : Communication asynchrone inter-threads.
				\item \texttt{QueryReceiver} : Réception des requêtes DNS.
				\item \texttt{QueryForwarder} : Transmission des requêtes DNS.
			\end{itemize}
		\end{itemize}
	\end{frame}

	% Choix d'Implémentation
	\begin{frame}
		\frametitle{Choix d'Implémentation Majeurs}
		\textbf{Préparation pour Firecracker}
		\begin{itemize}
			\item Définition de configurations de microVM type : \texttt{vm\_config.json} (template) et \texttt{vm\_test\_config.json} (pour tests).
			\begin{itemize}
				\item Spécification : image noyau, arguments de boot, rootfs, vCPUs, mémoire, réseau.
			\end{itemize}
			\item Scripts pour la création du rootfs, compilation du noyau, et téléchargement de Firecracker.
		\end{itemize}
	\end{frame}

	% Fonctionnalités Réalisées
    \section{Fonctionnalités Réalisées}
	\begin{frame}
		\frametitle{Fonctionnalités Réalisées}
		\begin{itemize}
			\item \textbf{Fondations du proxy/serveur DNS \texttt{microDNS} :}
			\begin{itemize}
				\item Réception et analyse basique des requêtes DNS.
				\item Mécanisme de transmission (forwarding) des requêtes DNS géré par \texttt{QueryForwarder}.
				\item Utilisation de files de messages (\texttt{MessageQueue}) pour une gestion asynchrone.
				\item Système de chargement de configuration dynamique (\texttt{ConfigLoader} via \texttt{microDNS.conf}).
			\end{itemize}
		\end{itemize}
	\end{frame}

		% Fonctionnalités Réalisées
	\begin{frame}
		\frametitle{Fonctionnalités Réalisées}
		\begin{itemize}
			\item \textbf{Configuration des microVM Firecracker :}
			\begin{itemize}
				\item Fichier \texttt{vm\_config.json} et \texttt{vm\_test\_config.json} pour spécifier les paramètres des microVMs.
				\item Capacité à lancer manuellement des microVMs Firecracker avec la configuration adéquate pour exécuter des fonctions.
			\end{itemize}
		\end{itemize}
	\end{frame}

	% État du Projet et Prochaines Étapes
    \section{État Actuel, Difficultés et Perspectives}
	\begin{frame}
		\frametitle{État Actuel, Difficultés et Perspectives}
		\textbf{Parties Incomplètes :}
		\begin{itemize}
			\item Intégration complète du lancement et de la gestion des microVM Firecracker par \texttt{microDNS} (via le composant \texttt{MicroVMManager}).
			\item Implémentation du communicateur \texttt{Vsock} pour la liaison microVM $\leftrightarrow$ hôte.
			\item Monitoring avancé et chaînage des fonctions.
		\end{itemize}
	\end{frame}

		% État du Projet et Prochaines Étapes
	\begin{frame}
		\frametitle{État Actuel, Difficultés et Perspectives}
		\textbf{Difficultés Rencontrées :}
		\begin{itemize}
			\item Complexité de l'intégration avec \texttt{systemd-resolved}.
			\item Mise en place de l'environnement de développement (macOS, virtualisation imbriquée).
			\item Compréhension initiale du sujet et de l'idéologie Firecracker.
		\end{itemize}
		\vspace{0.5em}
		\textbf{Améliorations Possibles :}
		\begin{itemize}
			\item Optimisation des performances DNS de \texttt{microDNS}.
			\item Utilisation du "jailer" de Firecracker et renforcement de l'isolation.
		\end{itemize}
	\end{frame}

	% Usage (Très bref)
    \section{Usage et Démonstration (État Actuel)}
	\begin{frame}
		\frametitle{Usage et Démonstration (État Actuel)}
		\textbf{Pour \texttt{microDNS} :}
		\begin{enumerate}
			\item Naviguer vers \texttt{dns/}.
			\item Compiler : \texttt{make microDNS}.
			\item Configurer l'environnement DNS (scripts) : \texttt{cd scripts/ \&\& ./microDNS.sh setup}.
			\item Lancer : \texttt{cd .. \&\& build/microDNS}.
		\end{enumerate}
	\end{frame}

	% Usage (Très bref)
	\begin{frame}
		\frametitle{Usage et Démonstration (État Actuel)}
		\textbf{Pour Firecracker (Tests Manuels) :}
		\begin{enumerate}
			\item Naviguer vers \texttt{firecracker/}.
			\item Préparer l'environnement : \texttt{make} (téléchargements, création rootfs, compilation noyau).
			\item Lancer une microVM interactive : \texttt{make test} (login: root, pas de mdp).
			\item \textit{Tester l'exécution de fonction :}
			\begin{itemize}
				\item Modifier \texttt{vm\_test\_config.json}, ajouter à \texttt{boot\_args} : \texttt{init=/init.sh -- ./test.sh}.
				\item Lancer avec \texttt{make test}.
				\item Accéder à \texttt{http://172.16.0.2:8000} dans un navigateur pour voir le résultat de \texttt{test.sh}.
			\end{itemize}
		\end{enumerate}
	\end{frame}

	% Conclusion
	\begin{frame}
		\frametitle{Conclusion et Questions}
		\begin{itemize}
			\item Les fondations DNS du projet sont établies et fonctionnelles.
			\item La configuration et le lancement manuel de microVM pour l'exécution de fonctions sont possibles.
			\item L'intégration complète et l'automatisation via \texttt{microDNS} constituent la principale prochaine étape.
		\end{itemize}
		\vfill
		\centering
		\textit{Merci pour votre attention !}
		\vspace{1em}
		Des questions ?
	\end{frame}

\end{document}
