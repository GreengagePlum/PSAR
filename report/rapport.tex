\RequirePackage[l2tabu, orthodox]{nag}
\documentclass[12pt]{article}
\usepackage[T1]{fontenc}
\usepackage[utf8]{inputenc}
\usepackage[french]{babel}
\usepackage{amsthm,amssymb,amsmath,xcolor}
\usepackage{setspace}
\doublespacing
\usepackage{geometry}
\geometry{
    a4paper,
    total={170mm,257mm},
}
\usepackage{graphicx}
\graphicspath{ {./} }
\usepackage{microtype}
\usepackage{todonotes}
\usepackage{hyperref}
\hypersetup{
    colorlinks=true,
    linkcolor=blue,
    filecolor=magenta,
    urlcolor=cyan,
}

\author{Efe ERKEN}
\date{\today}
\title{Rapport PSAR ``Orchestration de micro-vm via un resolver DNS''}

\begin{document}
\maketitle

\section{Difficultés}
\begin{center}
	\includegraphics[scale=0.2]{placeholder.png}
\end{center}

\section{Choix d'implémentation}
Tout d'abord, petit avertissement. J'ai fait le choix d'utiliser
\href{https://dbeaver.io/}{\texttt{DBeaver}} comme client SQL et environnement
de dévéloppement pour la première partie du projet. [...] en incluant le charactère
``\texttt{/}'' aux bons endroits... \\

\end{document}

